\documentclass{article}

\usepackage{booktabs}
\usepackage{threeparttable}
\usepackage[margin=1in]{geometry}
\usepackage{makecell}
\usepackage{amssymb} 
\usepackage[T1]{fontenc}
\usepackage[utf8]{inputenc}
\usepackage{amsmath,amssymb}
\usepackage{lmodern}
\usepackage[hidelinks]{hyperref}
\usepackage{caption}
\usepackage{placeins}

\captionsetup{
  font=small,
  labelfont=bf,
  skip=6pt
}

\setlength{\textfloatsep}{14pt}   % space between floats and text
\setlength{\floatsep}{12pt}       % space between two floats
\setlength{\intextsep}{12pt}       % space for in-text floats

\begin{document}

\begin{titlepage}
  \centering
  \vspace*{\fill}
  {\huge\bfseries FEII - Problem Set 1, 2025\par}
  \vspace{1.5cm}
  {\large Ivan Khalin\par}
  \vspace{0.2cm}
  {\large \href{mailto:ivan.khalin@unil.ch}{ivan.khalin@unil.ch}\par}
  \vspace{0.2cm}
  {\large \today\par}
  \vspace{1.25cm}
  {\large All code and data are available on GitHub:\par}
  {\large \url{https://github.com/vengveng/FEII}\par}
  \vspace*{\fill}
\end{titlepage}


\subsection*{Data and Variables}
This report discusses the replication of Table VIII from Drechsler, Savov and Schnabl (2017) “The Deposits Channel of Monetary Policy”, referred to hereafter as DSS

Additional data was downloaded from the FRED website, in particular the Federal Fund Target Rate (\texttt{DFEDTAR}). This target rate series was discontinued in December 2008, and was replaced by two new series: the upper and lower limit of the federal funds target range (\texttt{DFEDTARU} and \texttt{DFEDTARL} respectively). To construct the $FF$ variable, the midpoint of the upper and lower limit series was calculated and collated with the target rate series ending in 2008. The rate data was then resampled to quarterly by taking the rate of to the final day of the corresponding quarter.
% \vspace{10pt}

The assignment indicated the presence of duplicate observations at the cert-quarter level, cert being the FDIC bank identifier. Inspection showed that rows for the bank \texttt{FIRST FOUND BK} (\texttt{rssdid} 3637685) incorrectly shared the FDIC identifier (\texttt{cert} 58672) with \texttt{JEFFERSON BK OF FL}, creating overlapping cert-quarter entries. The error was corrected by reassigning the affected bank's certificate number to the correct 58647 for all rows with \texttt{rssdid} 3637685, after which no cert-quarter duplicates remained.
% \vspace{10pt}

The HHI data was merged onto the CR dataset by cert-dateq. After the merge, 6.72\% of the rows that matched successfully had missing \texttt{l1\_herfdepcty} values. These missing values affect 12{,}546 out of 13{,}182 total certificates in the sample (95.18\%). As discussed below, my regression sample was smaller than that of the authors, and the missing HHI data was one of the suspected culprits for this disparity. In an attempt to remedy this, all regressions were reestimated with the HHI values forward filled. The coefficient estimates of those regressions were incoherent with the original paper, and were thus not reported. In the final dataset, no forward filling of HHI was done.
% \vspace{10pt}

\subsection*{Results}
All regression variables were constructed exactly as described by DSS. Wholesale funding was defined as total liabilities (\texttt{liabilities}) minus total deposits (\texttt{deposits}). The below tables will all include the bank and quarter fixed effects, as well as a bank fixed effect interacted with a post-2008 indicator, unless otherwise specified. All coefficients represent the values of regressions where the column label is the dependent variable, and the exogenous regressor is the $\Delta FF_{t} \times \text{Bank HHI}$. Additionally, every row of the CR dataset was filtered to ensure that none of the entries are missing for any given row.

The first regression results (row None), reported in Table~\ref{tab:filtering}, have coefficients that are broadly consistent with DSS, with the exception of two coefficients in Panel~A. The first inconsistency is the regression of deposit spread in column~(2), where the estimate is small and statistically insignificant. The second inconsistency is in column~(5), the regression of wholesale funding, which exceeds the DSS estimates considerably. Further exploration of the replication code revealed that DSS applied 1\% winsorization to log changes in balance sheet components, and applied a ``growth'' filter to control for firms where asset growth exceeded 100\% quarter-over-quarter.\footnote{Both procedures follow the authors' Stata replication code. Winsorization of log changes in balance sheet components is implemented in \texttt{depositspread\_020117.do}, while the $>100\%$ quarter-on-quarter asset growth filter is applied in the construction of the call report sample in \texttt{banks\_020117.do}. These files are available in the DSS replication package on the Harvard Dataverse: \url{https://dataverse.harvard.edu/dataset.xhtml?persistentId=doi:10.7910/DVN/KHNXYJ}.}

Applying the DSS filter to the dataset brings all the coefficient estimates close in line to those found in the paper. In addition, the deposit spread estimates became significant at a 1\% level, and the wholesale funding coefficient shrinks considerably. Multiple regressions were performed to determine which of these filtering techniques drove the change in significance of the deposit spread, as shown in the different rows of Table~\ref{tab:filtering}. It turned out that winsorization was responsible, as evidenced by the ``Growth'' row in Panel~A, where the deposit spread coefficient is small and insignificant unlike in the ``Growth+Winsor'' row. In Panel~B, all coefficients with the exception of real estate loans were also closer to the original after the filtering was applied.

Interestingly, even for the unfiltered regression, the number of observations fell about sixty thousand short of DSS; the cause of this discrepancy has continued to elude me up until the submission of this report. In all likelihood, it is related to how the regressors are constructed by the authors. They most probably have more robust variable definitions and data treatment that results in fewer observations being dropped when they filter out for missing values.

\subsection*{Fixed Effects}
The fixed effects (FE) address potential omitted variable bias (OVB). The bank-level FE absorbs the heterogeneity across banks, controlling for time-invariant bank-specific characteristics, which effectively allows us to compare each bank to itself over time, rather than make comparisons across banks. The quarter FE absorbs aggregate shocks affecting all banks in any given quarter, so that at each time step the regressions compare banks within the same macroeconomic environment. Finally, the bank FE interacted with a post-2008 indicator allows for variation in bank-specific characteristics following the financial crisis. Since the sample extends well past 2008, the indicator is a reasonable addition; after the crisis, major regulatory changes that affected bank balance sheets were implemented, on top of any changes that the banks themselves may have privately chosen to undergo. Therefore, it is reasonable to expect that pooling bank-level FE across the pre- and post-2008 periods would conflate the two structural regimes.

The effects of omitting various FE are shown in Table~\ref{tab:x}.\footnote{The omission of bank and quarter FE will not be discussed as they are required for controlling OVB.} Running a regression without the post-2008 FE (second row) shows that on the liability side all coefficients shrink slightly in absolute terms with the exception of wholesale funding, for which the coefficient rises significantly. On the asset side, changes in coefficients are larger and more variable---total loans and C\&I loans lose significance entirely. This joint pattern of reduced but stable liability-side responses and larger, less precisely estimated asset-side responses when the post-2008 FE are removed suggests that these FE capture structural variation that is economically meaningful. It is consistent with the post-2008 bank FE absorbing a substantial share of the longer-run, bank-specific balance sheet adjustment after the crisis, so that excluding them shifts part of this structural break into the estimated response to Fed funds rate changes. Moreover, the asymmetry in how the coefficient estimates are impacted suggests that the post-crisis period features larger adjustments on the asset side.

\subsection*{Robustness Check}
Restricting the sample to the pre-2008 period, as reported in Table~\ref{tab:robust}, shows that the deposit spread channel remains intact. The coefficient is nearly identical to the full-sample estimate and equally significant, but other coefficients weaken considerably on both sides of the balance sheet. On the liability side, total deposits fall by more than half. Savings deposits lose significance, consistent with these accounts being sticky in normal times. Households hold savings accounts primarily for liquidity and convenience, so the difficulty and costs of switching reduce rate sensitivity. Time deposits remain strongly negative, reflecting their rate sensitivity even outside crisis episodes, as these accounts are held explicitly for investment purposes. Wholesale funding nearly doubles in magnitude, suggesting high-HHI banks relied more heavily on wholesale markets pre-crisis, though the estimate is noisier. On the asset side, the estimates weaken considerably: total assets contract by roughly one-third the baseline magnitude, and loans and securities lose statistical significance. This decline suggests that the balance sheet reallocation observed in the full sample is primarily a post-crisis phenomenon, likely driven by more stringent capital and liquidity requirements that forced banks to respond to rate hikes through visible contractions in specific asset types. Perhaps, before 2008 banks adjusted through less visible channels such as loan pricing, credit standards, or shifts between similar assets that did not produce large quarter-to-quarter changes in reported balance sheet totals.

Restricting the sample to the top 25\% of banks by assets yields coefficients broadly consistent with the full sample, though standard errors increase due to the smaller sample size (again, see Table~\ref{tab:robust}). Deposit spread, savings deposits, time deposits, and wholesale funding all retain their signs and significance. On the asset side, the securities response remains significant while loans lose precision, consistent with larger banks having more flexibility to adjust their securities portfolios. Restricting further to the top 10\% produces estimates that are statistically indistinguishable from zero for all specifications. This likely reflects two factors: first, the largest banks operate nationally and face less binding local deposit market power; second, the severely reduced sample size (under 50{,}000 observations) provides insufficient power to detect effects. The results suggest that while the deposits channel persists across most of the banking sector, it may be attenuated among the very largest institutions that enjoy more market power.

\clearpage

\begin{table}[!htbp]
\centering
\caption{Coefficient estimates under alternative filtering specifications}
\small
\renewcommand{\arraystretch}{1.15}
\setlength{\tabcolsep}{4pt}

\textbf{Panel A: Liabilities}\par\medskip
\begin{tabular*}{\textwidth}{@{\extracolsep{\fill}}lccccccc@{}}
   \toprule
   Filter & \makecell[c]{$\Delta$Total\\deposits} & \makecell[c]{$\Delta$Deposit\\spread} & \makecell[c]{$\Delta$Savings\\deposits} & \makecell[c]{$\Delta$Time\\deposits} & \makecell[c]{$\Delta$Wholesale\\funding} & \makecell[c]{$\Delta$Total\\liabilities} & Obs.\\
          & (1) & (2) & (3) & (4) & (5) & (6) & \\
   \midrule
   \multicolumn{8}{c}{$\Delta FF_t \times$ Bank HHI}\\
   \midrule
   None & \makecell{-1.562$^{***}$ \\ (0.179)} & \makecell{0.026 \\ (0.021)} & \makecell{-1.399$^{***}$ \\ (0.294)} & \makecell{-2.432$^{***}$ \\ (0.260)} & \makecell{2.793$^{***}$ \\ (1.042)} & \makecell{-1.352$^{***}$ \\ (0.173)} & 496,257\\
   Growth & \makecell{-1.536$^{***}$ \\ (0.162)} & \makecell{0.029 \\ (0.021)} & \makecell{-1.342$^{***}$ \\ (0.284)} & \makecell{-2.423$^{***}$ \\ (0.245)} & \makecell{2.766$^{***}$ \\ (1.036)} & \makecell{-1.334$^{***}$ \\ (0.154)} & 495,643\\
   \makecell{Growth+\\Winsor} & \makecell{-1.488$^{***}$ \\ (0.145)} & \makecell{0.063$^{***}$ \\ (0.006)} & \makecell{-1.209$^{***}$ \\ (0.243)} & \makecell{-2.186$^{***}$ \\ (0.213)} & \makecell{2.637$^{***}$ \\ (0.954)} & \makecell{-1.285$^{***}$ \\ (0.139)} & 495,643\\
   \bottomrule
\end{tabular*}

\medskip
\textbf{Panel B: Assets}\par\medskip
\begin{tabular*}{\textwidth}{@{\extracolsep{\fill}}lccccccc@{}}
   \toprule
   Filter & \makecell[c]{$\Delta$Total\\assets} & \makecell[c]{$\Delta$Cash} & \makecell[c]{$\Delta$Securities} & \makecell[c]{$\Delta$Total\\loans} & \makecell[c]{$\Delta$RE\\loans} & \makecell[c]{$\Delta$C\&I\\loans} & Obs.\\
          & (1) & (2) & (3) & (4) & (5) & (6) & \\
   \midrule
   \multicolumn{8}{c}{$\Delta FF_t \times$ Bank HHI}\\
   \midrule
   None & \makecell{-1.240$^{***}$ \\ (0.157)} & \makecell{-2.148$^{***}$ \\ (0.732)} & \makecell{-1.084$^{**}$ \\ (0.507)} & \makecell{-0.532$^{***}$ \\ (0.189)} & \makecell{-0.829$^{***}$ \\ (0.262)} & \makecell{-1.134$^{***}$ \\ (0.421)} & 496,257\\
   \makecell{Growth+\\Winsor} & \makecell{-1.205$^{***}$ \\ (0.122)} & \makecell{-2.348$^{***}$ \\ (0.672)} & \makecell{-0.938$^{***}$ \\ (0.333)} & \makecell{-0.484$^{***}$ \\ (0.151)} & \makecell{-0.857$^{***}$ \\ (0.190)} & \makecell{-0.896$^{***}$ \\ (0.341)} & 495,643\\
   \bottomrule
\end{tabular*}
\caption*{The table reports coefficients on $\Delta FF_t \times$ Bank HHI from regressions of log changes in balance sheet items (columns) on the interaction of the Fed funds shock and bank-level market concentration. 
All specifications include bank fixed effects, quarter fixed effects, and bank fixed effects interacted with a post-2008 indicator. 
The row ``None'' uses the full sample without additional filtering; ``Growth'' drops bank–quarters in which total assets grow by more than 100\% quarter-on-quarter; ``Growth+Winsor'' additionally winsorizes the independent variables at the 1st and 99th percentiles. 
Standard errors (in parentheses) are clustered at the bank (\texttt{rssdid}) level.}
\label{tab:filtering}
\end{table}


\FloatBarrier

\begin{table}[!htbp]
\centering
\caption{Sensitivity to fixed-effects specification}
\small
\renewcommand{\arraystretch}{1.15}
\setlength{\tabcolsep}{4pt}

\textbf{Panel A: Liabilities}\par\medskip
\begin{tabular*}{\textwidth}{@{\extracolsep{\fill}}ccccccccc@{}}
   \toprule
   \makecell[c]{Bank\\f.e.} & \makecell[c]{Quarter\\f.e.} & \makecell[c]{Bank $\times$\\2008 f.e.} & \makecell[c]{$\Delta$Total\\deposits} & \makecell[c]{$\Delta$Deposit\\spread} & \makecell[c]{$\Delta$Savings\\deposits} & \makecell[c]{$\Delta$Time\\deposits} & \makecell[c]{$\Delta$Wholesale\\funding} & \makecell[c]{$\Delta$Total\\liabilities}\\
           &               &                    & (1) & (2) & (3) & (4) & (5) & (6)\\
   \midrule
   \multicolumn{9}{c}{$\Delta FF_t \times$ Bank HHI}\\
   \midrule
   Y & Y & Y & \makecell{-1.488$^{***}$ \\ (0.145)} & \makecell{0.063$^{***}$ \\ (0.006)} & \makecell{-1.209$^{***}$ \\ (0.243)} & \makecell{-2.186$^{***}$ \\ (0.213)} & \makecell{2.637$^{***}$ \\ (0.954)} & \makecell{-1.285$^{***}$ \\ (0.139)}\\
   Y & Y &  & \makecell{-1.141$^{***}$ \\ (0.144)} & \makecell{0.058$^{***}$ \\ (0.006)} & \makecell{-1.007$^{***}$ \\ (0.241)} & \makecell{-1.696$^{***}$ \\ (0.212)} & \makecell{3.488$^{***}$ \\ (0.959)} & \makecell{-0.909$^{***}$ \\ (0.138)}\\
   \bottomrule
\end{tabular*}

\medskip
\textbf{Panel B: Assets}\par\medskip
\begin{tabular*}{\textwidth}{@{\extracolsep{\fill}}ccccccccc@{}}
   \toprule
   \makecell[c]{Bank\\f.e.} & \makecell[c]{Quarter\\f.e.} & \makecell[c]{Bank $\times$\\2008 f.e.} & \makecell[c]{$\Delta$Total\\assets} & \makecell[c]{$\Delta$Cash} & \makecell[c]{$\Delta$Securities} & \makecell[c]{$\Delta$Total\\loans} & \makecell[c]{$\Delta$RE\\loans} & \makecell[c]{$\Delta$C\&I\\loans}\\
           &               &                    & (1) & (2) & (3) & (4) & (5) & (6)\\
   \midrule
   \multicolumn{9}{c}{$\Delta FF_t \times$ Bank HHI}\\
   \midrule
   Y & Y & Y & \makecell{-1.205$^{***}$ \\ (0.122)} & \makecell{-2.348$^{***}$ \\ (0.672)} & \makecell{-0.938$^{***}$ \\ (0.333)} & \makecell{-0.484$^{***}$ \\ (0.151)} & \makecell{-0.857$^{***}$ \\ (0.190)} & \makecell{-0.896$^{***}$ \\ (0.341)}\\
   Y & Y &  & \makecell{-0.907$^{***}$ \\ (0.121)} & \makecell{-2.351$^{***}$ \\ (0.679)} & \makecell{-0.697$^{**}$ \\ (0.334)} & \makecell{-0.042 \\ (0.151)} & \makecell{-0.385$^{**}$ \\ (0.190)} & \makecell{-0.472 \\ (0.339)}\\
   \bottomrule
\end{tabular*}
\caption*{Sensitivity of the estimated $\Delta FF_t \times$ Bank HHI coefficients to the fixed-effects specification. 
All regressions use the same baseline sample, filters, and variables as in Table~\ref{tab:filtering} (Growth+Winsor, full sample). 
Rows differ only in which fixed effects are included: the baseline row retains bank, quarter, and bank$\times$post-2008 fixed effects, while the alternative drops the bank$\times$post-2008 term.}
\label{tab:x}
\end{table}

\FloatBarrier

\begin{table}[!htbp]
\centering
\caption{Robustness checks}
\small
\renewcommand{\arraystretch}{1.15}
\setlength{\tabcolsep}{4pt}

\textbf{Panel A: Liabilities}\par\medskip
\begin{tabular*}{\textwidth}{@{\extracolsep{\fill}}lccccccc@{}}
   \toprule
   Sample & \makecell[c]{$\Delta$Total\\deposits} & \makecell[c]{$\Delta$Deposit\\spread} & \makecell[c]{$\Delta$Savings\\deposits} & \makecell[c]{$\Delta$Time\\deposits} & \makecell[c]{$\Delta$Wholesale\\funding} & \makecell[c]{$\Delta$Total\\liabilities} & Obs.\\
          & (1) & (2) & (3) & (4) & (5) & (6) & \\
   \midrule
   \multicolumn{8}{c}{$\Delta FF_t \times$ Bank HHI}\\
   \midrule
   Full sample & \makecell{-1.488$^{***}$ \\ (0.145)} & \makecell{0.063$^{***}$ \\ (0.006)} & \makecell{-1.209$^{***}$ \\ (0.243)} & \makecell{-2.186$^{***}$ \\ (0.213)} & \makecell{2.637$^{***}$ \\ (0.954)} & \makecell{-1.285$^{***}$ \\ (0.139)} & 495,643\\
   Pre-2008 & \makecell{-0.680$^{***}$ \\ (0.158)} & \makecell{0.065$^{***}$ \\ (0.008)} & \makecell{0.423 \\ (0.274)} & \makecell{-2.129$^{***}$ \\ (0.257)} & \makecell{4.980$^{***}$ \\ (1.120)} & \makecell{-0.470$^{***}$ \\ (0.153)} & 417,534\\
   Top 25\% assets & \makecell{-1.558$^{***}$ \\ (0.478)} & \makecell{0.043$^{**}$ \\ (0.019)} & \makecell{-1.691$^{**}$ \\ (0.702)} & \makecell{-2.581$^{***}$ \\ (0.625)} & \makecell{2.349 \\ (1.908)} & \makecell{-1.263$^{***}$ \\ (0.436)} & 132,561\\
   Top 10\% assets & \makecell{0.163 \\ (0.716)} & \makecell{0.046 \\ (0.044)} & \makecell{-0.221 \\ (1.264)} & \makecell{-0.110 \\ (1.110)} & \makecell{2.213 \\ (3.165)} & \makecell{-0.163 \\ (0.673)} & 48,211\\
   \bottomrule
\end{tabular*}

\medskip
\textbf{Panel B: Assets}\par\medskip
\begin{tabular*}{\textwidth}{@{\extracolsep{\fill}}lccccccc@{}}
   \toprule
   Sample & \makecell[c]{$\Delta$Total\\assets} & \makecell[c]{$\Delta$Cash} & \makecell[c]{$\Delta$Securities} & \makecell[c]{$\Delta$Total\\loans} & \makecell[c]{$\Delta$RE\\loans} & \makecell[c]{$\Delta$C\&I\\loans} & Obs.\\
          & (1) & (2) & (3) & (4) & (5) & (6) & \\
   \midrule
   \multicolumn{8}{c}{$\Delta FF_t \times$ Bank HHI}\\
   \midrule
   Full sample & \makecell{-1.205$^{***}$ \\ (0.122)} & \makecell{-2.348$^{***}$ \\ (0.672)} & \makecell{-0.938$^{***}$ \\ (0.333)} & \makecell{-0.484$^{***}$ \\ (0.151)} & \makecell{-0.857$^{***}$ \\ (0.190)} & \makecell{-0.896$^{***}$ \\ (0.341)} & 495,643\\
   Pre-2008 & \makecell{-0.465$^{***}$ \\ (0.136)} & \makecell{-3.108$^{***}$ \\ (0.637)} & \makecell{0.156 \\ (0.372)} & \makecell{0.217 \\ (0.196)} & \makecell{0.127 \\ (0.245)} & \makecell{-0.105 \\ (0.417)} & 417,534\\
   Top 25\% assets & \makecell{-1.238$^{***}$ \\ (0.385)} & \makecell{1.422 \\ (1.559)} & \makecell{-2.160$^{***}$ \\ (0.825)} & \makecell{-0.561 \\ (0.394)} & \makecell{-0.304 \\ (0.470)} & \makecell{-0.171 \\ (0.677)} & 132,561\\
   Top 10\% assets & \makecell{-0.212 \\ (0.643)} & \makecell{3.672 \\ (2.847)} & \makecell{-1.319 \\ (1.737)} & \makecell{0.595 \\ (0.818)} & \makecell{0.846 \\ (0.861)} & \makecell{-0.737 \\ (1.397)} & 48,211\\
   \bottomrule
\end{tabular*}
\caption*{Robustness of the $\Delta FF_t \times$ Bank HHI coefficients across alternative samples. 
Each row reports results from the same baseline specification and filtering as in Table~\ref{tab:filtering} (Growth+Winsor, with the full fixed-effects structure from Table~\ref{tab:x}), but estimated on a different sample: the full sample, a pre-2008 subsample, and subsamples restricted to banks in the top 25\% and top 10\% of the asset distribution. }
\label{tab:robust}
\end{table}

\FloatBarrier

\end{document}